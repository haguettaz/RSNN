\documentclass{article}
\usepackage[a4paper, total={6in, 9in}]{geometry}
\usepackage[utf8]{inputenc}
\usepackage{research}
\usepackage[english]{babel}
\usepackage{tikz} 


\title{On sampling firing signals}
\author{Hugo Aguettaz}
\begin{document}
\maketitle
    \tikzstyle{st0} = [draw, fill=black, rectangle, minimum width=3mm, minimum height = 3mm]
    \tikzstyle{st1} = [draw, rectangle, minimum width=5mm, minimum height=5mm]
    \tikzstyle{st2} = [draw, rectangle, inner sep=0pt, minimum height=0mm, minimum width=0mm] 

    In this paper, we present a backward filtering forward sampling algorithm to uniformly sample sequences on the set of 
    firing sequences of length $n \in \Integers_{\geq 0}$ and with a refratory period $T_r \in \Integers_{\geq 0}$.

    We recall the mathematical definition of this set :
    \begin{equation}
        \mathcal{Y}_{T_r}^n = \left\{ (y_1, \dots, y_n) \in \{0, 1\}^n : \sum_{m=0}^{T_r} y_{(k+m \mod{n}) + 1} \in \{0, 1\}, k \in \Nk{n} \right\}.
    \end{equation}

    We would like to create samples $(x_1, \dots, x_n)$ such that:
    \begin{equation}  
      p (x_1, \dots, x_n) =
      \left\{
        \begin{array}{ll}
          \left(\ell_{T_r}^n\right)^{-1} & \mbox{if } (x_1, \dots, x_n) \in \mathcal{Y}_{T_r}^{n}\\
          0 & \mbox{otherwise} \\
        \end{array}
      \right..
    \end{equation}

    First of all, notice that if $n \leq 2 T_r$, the only valid sequence is the all-zeros one and if $n = 2 T_r + 1$, 
    a valid sequence contains at most one spike. In these two cases, we can easily sample a valid sequence uniformly. From now on,
    we consider the non-trivial cases, and we thus assume $n > 2 T_r + 1$.
  

    Let $Z_k = (X_{k-T_r+1}, \dots, X_{k})$ represents random sequences of length $T_r$ with $k \in \{1, \dots, n\}$ and where
    $X_i, i \in \{k-T_r + 1, \dots , k\}$ takes value in $\{0,1\}$. 
    Obviously, this sequence is a Markov chain. Using the constraint functions
    \begin{equation}  
      g_{k-1,k}(z_{k-1}, z_{k}) =
      \left\{
        \begin{array}{ll}
          1 & \mbox{if } (x_{k-T_r}, \dots, x_{k}) \in \mathcal{Y}_{T_r}^{T_r + 1}\\
          0 & \mbox{otherwise} \\
        \end{array}
      \right.,
    \end{equation}
    we represent the construction of a valid sequence in $\mathcal{Y}_{T_r}^n$ as a factor graph with loop (\Cref{fig:factor_graph}).
    \begin{figure}[!ht]
      \centering
      \begin{tikzpicture}[node distance={20mm}] 
      \node (1){}; 
      \node[st1] (2) [right of=1] [label=below:$g_{n,1}$] {}; 
      \node[st1] (3) [right of=2] {}; 
      \node (4) [right of=3] {$\dots$}; 
      \node[st1] (5) [right of=4] {}; 
      \node[st1] (6) [right of=5] [label=below:$g_{n-1,n}$] {}; 
      \node (7) [right of=6] {}; 
      \draw (1) -- node[above] {$Z_{n}$} (2);
      \draw (2) -- node[above] {$Z_1$} (3);
      \draw (3) -- (4);
      \draw (4) -- (5);
      \draw (5) -- node[above] {$Z_{n-1}$} (6);
      \draw (6) -- node[above] {$Z_{n}$} (7);
      \end{tikzpicture}
      \caption{Factor graph} \label{fig:factor_graph}
    \end{figure}
    The sampling algorithm consists in iteratively constructing a sequence from this factor graph, in three steps:
    \begin{enumerate}
      \item Loop removing (c.f. \Cref{sec:samplez})
      \item Backward filtering (c.f. \Cref{sec:backward})
      \item Forward sampling (c.f. \Cref{sec:forward})
    \end{enumerate}
    
    \section{Loop-free factor graph} \label{sec:samplez}
    Fixing $Z_n$, we can transform the factor graph in \Cref{fig:factor_graph} into the loop-free 
    factor graph in \Cref{fig:factor_graph_fix}. 

    \begin{figure}[!ht]
      \centering
      \begin{tikzpicture}[node distance={20mm}] 
      \node[st0] (1) [label=below:$\breve{z}_{n}$] {}; 
      \node[st1] (2) [right of=1] [label=below:$g_{n,1}$] {}; 
      \node[st1] (3) [right of=2] {}; 
      \node (4) [right of=3] {$\dots$}; 
      \node[st1] (5) [right of=4] {}; 
      \node[st1] (6) [right of=5] [label=below:$g_{n-1,n}$] {}; 
      \node[st0] (7) [right of=6] [label=below:$\breve{z}_{n}$] {}; 
      \draw (1) -- node[above] {$Z_{n}$} (2);
      \draw (2) -- node[above] {$Z_1$} (3);
      \draw (3) -- (4);
      \draw (4) -- (5);
      \draw (5) -- node[above] {$Z_{n-1}$} (6);
      \draw (6) -- node[above] {$Z_{n}$} (7);
      \end{tikzpicture}
      \caption{Factor graph with fixed $Z_n = \breve{z}_n$} \label{fig:factor_graph_fix}
    \end{figure}

    To uniformly sample $z_n = (x_{n-T_r + 1}, \dots, x_n)$ it suffices to notice that it contains at most one spike.
    If $z_n$ is all-zeros, then we have 

    \begin{equation*}
      \underbrace{x_1 \cdots x_{n-T_r}}_{\in \tilde{\mathcal{Y}}_{T_r}^{n-T_r}} \overbrace{0 \cdots 0}^{T_r},
    \end{equation*}
    and there are $\tilde{\ell}_{T_r}^{n-T_r}$ such sequences. 

    If $z_n$ contains exactly one spike, then we have 
    \begin{equation*}
      \overbrace{0 \cdots 0}^{T_r - k_2} \underbrace{x_{T_r - k_2 + 1} \cdots x_{n-2T_r + k_1 - 1}}_{\in \tilde{\mathcal{Y}}_{T_r}^{n-2T_r-1}} 
      \overbrace{0 \cdots 0}^{T_r - k_1}
      \overbrace{0 \cdots 0}^{k_1} 
      1 
      \overbrace{0 \cdots 0}^{k_2},
    \end{equation*}
    with $k_1 + k_2 = T_r - 1, k_1, k_2 \geq 0$, and there are $\tilde{\ell}_{T_r}^{n-2T_r-1}$ such sequences. 

    As a consequence, we can simply sample $z_n$ according to 

    \begin{equation}  
      p(z_{n}) = p(x_{n-T_r + 1}, \dots, x_n) = 
      \left\{
        \begin{array}{ll}
          \frac{\tilde{\ell}_{T_r}^{n-T_r}}{\tilde{\ell}_{T_r}^{n-T_r} + \tilde{\ell}_{T_r}^{n-2T_r-1}} & \mbox{if } \sum_{k=n-T_r+1}^n x_{k} = 0 \\
          \frac{\tilde{\ell}_{T_r}^{n-2T_r-1}}{\tilde{\ell}_{T_r}^{n-T_r} + \tilde{\ell}_{T_r}^{n-2T_r-1}} & \mbox{if } \sum_{k=n-T_r+1}^n x_{k} = 1 \\
          0 & \mbox{otherwise} \\
        \end{array}
      \right..
    \end{equation}


    \section{Backward filtering} \label{sec:backward}

    \begin{figure}[!ht]
      \centering
      \begin{tikzpicture}[node distance={20mm}] 
      \node[st0] (1) [label=below:$\breve{z}_{n}$] {}; 
      \node[st1] (2) [right of=1] [label=below:$g_{n,1}$] {}; 
      \node[st1] (3) [right of=2] {}; 
      \node (4) [right of=3] {$\dots$}; 
      \node[st1] (5) [right of=4] {}; 
      \node[st1] (6) [right of=5] [label=below:$g_{n-1,n}$] {}; 
      \node[st0] (7) [right of=6] [label=below:$\breve{z}_{n}$] {}; 
      \draw[<-, thick, red] (1) -- node[above] {$\msgb{\mu}{Z_n}$} (2);
      \draw[<-, thick, red] (2) -- node[above] {$\msgb{\mu}{Z_1}$} (3);
      \draw (3) -- (4);
      \draw (4) -- (5);
      \draw[<-, thick, red] (5) -- node[above] {$\msgb{\mu}{Z_{n-1}}$} (6);
      \draw[<-, thick, red] (6) -- node[above] {$\msgb{\mu}{Z_n}'$} (7);
      \end{tikzpicture}
      \caption{Backward filtering} \label{fig:backward}
    \end{figure}
  
    Once, $z_n$ is fixed, we can compute the backward messages $\msgb{\mu}{Z_{k}}$ by sum-product message passing as illustrated in 
    \Cref{fig:backward}. Starting with the message 
    \begin{equation}
      \msgb{\mu}{Z_{n}}'(z_n) = \left\{
        \begin{array}{ll}
          1 & \mbox{if } z_n = \breve{z}_n\\
          0 & \mbox{otherwise} \\
        \end{array}
      \right.,
    \end{equation}
    we can recursively compute all backward messages, from right to left using:
    \begin{equation}
      \msgb{\mu}{Z_{k-1}}(z_{k-1}) = \sum_{z_{k}} g_{k-1, k}(z_{k-1}, z_{k}) \msgb{\mu}{Z_{k}}(z_{k}), \quad k \in \{1, \dots, n\},
    \end{equation}
    with $z_0 = z_n$.
    
    It can also be expressed in matrix form as
    \begin{equation}
      \mat{\msgb{\mu}{Z_{k-1}}} = \mat{\msgb{\mu}{Z_{k}}} \mat{A}
    \end{equation}
    with 
    \begin{equation}
      \left\{\mat{\msgb{\mu}{Z_{k}}}\right\}_{i_{z_k}} = \msgb{\mu}{Z_{k}}(z_{k})
    \end{equation}
    for $i_{z_k} \in \Nk{T_r + 1}$ and
    \begin{equation}
      z_k = \vec{0} \mapsto i_{z_k} = 1
    \end{equation}
    \begin{equation}
      z_k = \vec{e_{i}} \mapsto i_{z_k} = i + 1
    \end{equation}

    and with
    \begin{equation}
      \mat{A} = 
      \begin{bmatrix}
        1 & 1 & 0 & \cdots & 0 \\
        0 & 0 & 1 & \ddots & \vdots \\
        \vdots & \vdots & \ddots & \ddots & 0 \\
        0 & 0 & \cdots & 0 & 1 \\
        1 & 0 & \cdots & 0 & 0 \\
      \end{bmatrix} \in \{0, 1\}^{T_r + 1 \times T_r + 1}.
    \end{equation}

    \section{Forward sampling} \label{sec:forward}

    Using backward messages, we can sample by forward sampling as shown in \Cref{fig:forward}. We sample $z_k$ for $k \in \{1, \dots, n-1\}$ according to:
    \begin{align}
      p(z_k | z_{k-1}) & = \frac{p(z_k, z_{k-1})}{p(z_{k-1})} \\
      & = \frac{g_{k-1,k}(z_{k-1}, z_{k}) \msgf{\mu}{Z_{k-1}}(z_{k-1}) \msgb{\mu}{Z_{k}}(z_{k})}{\msgf{\mu}{Z_{k-1}}(z_{k-1}) \msgb{\mu}{Z_{k-1}}(z_{k-1})} \\
      & = \frac{g_{k-1,k}(z_{k-1}, z_{k}) \msgb{\mu}{Z_{k}}(z_{k})}{\msgb{\mu}{Z_{k-1}}(z_{k-1})},
    \end{align}
    with $z_{0} = z_n$.

    Again, this can be expressed in the matrix form
    \begin{equation}
      \mat{p_{Z_{k}|Z_{k-1}}} = \left\{\mat{\msgb{\mu}{Z_{k-1}}}\right\}_{i_{z_{k-1}}}^{-1} \mat{\msgb{\mu}{Z_{k}}} \mat{A}_{:, i_{\breve{z}_{k-1}}}.
    \end{equation}

    \begin{figure}[!ht]
      \centering
      \begin{tikzpicture}[node distance={15mm}, main/.style = {draw, rectangle, minimum width=5mm, minimum height = 5mm}] 
      \node[st0] (1) [label=below:$\breve{z}_{n}$] {}; 
      \node[st2] (12) [right of=1] {}; 
      \node[st1] (2) [right of=12] [label=below:$g_{n,1}$] {}; 
      \node[st2] (23) [right of=2] {}; 
      \node[st1] (3) [right of=23] {}; 
      \node (4) [right of=3] {$\dots$}; 
      \node[st1] (5) [right of=4] {}; 
      \node[st2] (56) [right of=5] {}; 
      \node[st1] (6) [right of=56] [label=below:$g_{n-1,n}$] {}; 
      \draw[->, thick, blue] (1) -- node[above] {$\msgf{\mu}{Z_n}$} (12);
      \draw[<-, thick, red] (12) -- node[above] {$\msgb{\mu}{Z_n}$} (2);      
      \draw[->, thick, blue] (2) -- node[above] {$\msgf{\mu}{Z_1}$} (23);
      \draw[<-, thick, red] (23) -- node[above] {$\msgb{\mu}{Z_1}$} (3);
      \draw (3) -- (4);
      \draw (4) -- (5);
      \draw[->, thick, blue] (5) -- node[above] {$\msgf{\mu}{Z_{n-1}}$} (56);
      \draw[<-, thick, red] (56) -- node[above] {$\msgb{\mu}{Z_{n-1}}$} (6);      
      \end{tikzpicture}
      \caption{Forward sampling} \label{fig:forward}
    \end{figure}
    
    

    %----------------------------------------------------------------------------------------
    \newpage
    \appendix



\end{document}

